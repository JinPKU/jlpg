%!TEX program = xelatex

\documentclass[a4paper]{article}
%\documentclass{amsart}
\usepackage[breaklinks,linkcolor=black,citecolor=black,urlcolor=black]
{hyperref}
\usepackage{amsmath}
\usepackage{amssymb}
\usepackage{amsthm}
\usepackage{amsfonts}
\usepackage{booktabs}
\usepackage{graphicx}
\usepackage{latexsym}
\usepackage{mathrsfs}
\usepackage[ruled]{algorithm2e}
\usepackage{subfigure}

%\usepackage{algorithm}
%\usepackage{algorithmic}
\usepackage{algpseudocode}
\usepackage{enumerate}
\usepackage{bm}
\usepackage{listings}
\usepackage{tikz}
\usetikzlibrary{arrows}
\usepackage[colorinlistoftodos]{todonotes}
\usepackage[a4paper,left=2.5cm, right=2.5cm, top=2.5cm, 
bottom=2.5cm]{geometry}
\usepackage{xeCJK}

%\newfontfamily{\con}{Consolas} 

%\lstset{language=Matlab}

\newcommand{\norm}[1]{\left\Vert#1\right\Vert}
\newtheorem{definition}{Definition}
\newtheorem{theorem}{Theorem}
\newtheorem{remark}{Remark}
\newtheorem{lemma}{Lemma}
\newtheorem{example}{Example}
\newtheorem{assumption}{Assumption}
\newtheorem{proposition}{Proposition}
\newtheorem{experiment}{Experiment}

%\numberwithin{equation}{section}
\newcommand{\ip}[2]{\left\langle{#1},{#2}\right\rangle} % inner-prod
\newcommand{\mr}{\mathbb{R}}
\newcommand{\mh}{\mathcal{H}}
\newcommand{\prox}{\mathrm{prox}}
\newcommand{\sign}{\mathrm{sign}}
\newcommand{\proj}{\mathrm{Proj}}
\newcommand{\mi}{\mathbf{1}}

%\graphicspath{{../figures/}}



%--------------------------------------------------------------------%
\title{Software implementation for the proximal gradient methods}
\author{}
\date{\today}

\begin{document}
\maketitle
\tableofcontents

\section{Introduction}
\label{sec:intro}
In this project, we consider the following composite optimization 
problem
\begin{equation}
  \min\limits_{x} F(x) = f(x) + h(x),
\end{equation}
where $f(x)$ is differentiable and $h(x)$ is a function whose 
proximal operator is easily available.
The proximal operator of $h(x)$ is defined as
\begin{displaymath}
  \prox_h(x) = \mathop{\arg\min}\limits_{u} h(u) + \frac{1}{2} 
  \norm{u - x}^2.
\end{displaymath}
Starting from a suitable initial point $x^0$, the proximal gradient 
method is performed as
\begin{displaymath}
  x^{k+1} = \prox_{t_k h} (x^k - t_k \nabla f(x^k)),
\end{displaymath}
where $t_k$ is a chosen step size.


% description of algorithms & interpretation of implementation?
\section{Algorithms}
\label{sec:alg}

\subsection{Strategies for choosing the step size}






\section{Several scenarios}
\label{sec:cal}

\subsection{Scenarios of $f(x)$}
\label{sec:cal_f}
In this part, we list several typical scenarios of $f(x)$ that are 
supported in our codes. We give their gradient explicitly.

\begin{itemize}
  \item {Least squares in vector form:} $f(x) = \frac{1}{2} 
  \norm{Ax - b}_2^2$, where $A \in \mr^{m \times n}$, $x \in \mr^n$, 
  $b 
  \in \mr^m$. The gradient of $f(x)$ is given by %\footnote{In this 
  %report, we write the gradient in the same shape as $x$.}
  \begin{displaymath}
    \nabla f(x) = A^T (A x - b).
  \end{displaymath}
  
  \item {Least squares in matrix form:} $f(x) = \frac{1}{2} 
  \norm{Ax - b}_F^2$, where $A \in \mr^{m \times n}$, $x \in \mr^{n 
    \times l}$, $b \in \mr^{m \times l}$. The gradient of $f(x)$ is 
  given 
  by
  \begin{displaymath}
    \nabla f(x) = A^T  (A x - b).
  \end{displaymath}
  
  \item {Logistic regression:} 
  \begin{displaymath}
    f(x) = \frac{1}{m} \sum_{i = 1}^{m} \log(1 + \exp(- b_i a_i^T x)),
  \end{displaymath}
  where $x \in \mr^n$, $A = (a_1, \ldots, a_m) \in \mr^{n \times m}$, 
  $a_i = (a_{1,i}, \ldots, a_{n,i})^T \in \mr^n$,
  $b = (b_1, \ldots, b_m)^T \in \mr^m$. The gradient of $f(x)$ is 
  given 
  by
  \begin{displaymath}
    \nabla f(x) = - \frac{1}{m} \sum_{i = 1}^{m} \frac{b_i 
      a_i}{\exp(b_i a_i^T x) + 1}.
  \end{displaymath}
\end{itemize}



\subsection{Scenarios of $h(x)$}
\label{sec:cal_h}
In this part, we list several typical scenarios of $h(x)$ that are 
supported in our codes. We give the proximal operator of $\mu h(x)$ 
explicitly.

\begin{itemize}
  \item {Vector $\ell_0$-norm:} $h(x) = \norm{x}_0 = \# \{i:x_i 
  \ne 0\}$, where $x \in \mr^n$. The proximal operator of $\mu h(x)$ 
  is 
  given by
  \begin{displaymath}
    \prox_{\mu h}(x) = \mh_{\mu}(x_1) \times \ldots \times 
    \mh_{\mu}(x_n),
  \end{displaymath}
  where $\mh_{\mu}(s)$ is given by
  \begin{displaymath}
    \mh_{\mu}(s) = 
    \begin{cases}
      \{0\},    & |s| < \sqrt{2\mu}, \\
      \{0, s\}, & |s| = \sqrt{2\mu}, \\
      \{s\},    & |s| > \sqrt{2\mu}.
    \end{cases}
  \end{displaymath}
  
  \item {Vector $\ell_1$-norm:} $h(x) = \norm{x}_1 = 
  \sum_{i=1}^{n} |x_i|$, where $x \in \mr^n$. The proximal operator 
  of 
  $\mu h(x)$ is given by
  \begin{displaymath}
    \prox_{\mu h}(x) = (\mh_{\mu}(x_1), \ldots, \mh_{\mu}(x_n))^T,
  \end{displaymath}
  where $\mh_{\mu}(s)$ is given by
  \begin{displaymath}
    \mh_{\mu}(s) = \sign(s) \max\{ |s| - \mu, 0 \}.
  \end{displaymath}
  
  \item {Vector $\ell_2$-norm:} $h(x) = \norm{x}_2 = 
  \left( \sum_{i=1}^{n} |x_i|^2 \right)^{\frac{1}{2}}$, where $x \in 
  \mr^n$. The proximal operator of $\mu h(x)$ is given by
  \begin{displaymath}
    \prox_{\mu h}(x) = 
    \begin{cases}
      \frac{\max\{ \norm{x}_2 - \mu, 0 \}}{\norm{x}_2} x, & x \ne 0, 
      \\
      0, & x = 0.
    \end{cases}
  \end{displaymath}
  
  
  \item {Vector $\ell_\infty$-norm:} $h(x) = \norm{x}_\infty = 
  \max\{ |x_i|: i = 1, 2, \ldots, n \}$, where $x \in 
  \mr^n$. By Moreau's decomposition, the proximal operator of $\mu 
  h(x)$ is given by
  \begin{displaymath}
    \begin{split}
      \prox_{\mu h}(x) & = x - \mu \; \prox_{\mu^{-1} h^\star} \left( 
      \frac{x}{\mu} \right) \\
      & = x - \mu \; \proj_{B_{\norm{\cdot}_1}(1)} \left( 
      \frac{x}{\mu} \right),
    \end{split}
  \end{displaymath}
  where $h^\star(y) = \mi_{B_{\norm{\cdot}_1}(1)}(y)$, $\mi_C(y)$ is 
  the indicator function defined by
  \begin{displaymath}
    \mi_C(y) = 
    \begin{cases}
      0, & y \in C, \\
      +\infty, & y \notin C,
    \end{cases}
  \end{displaymath}
  $B_{\norm{\cdot}}(r)$ is the norm ball defined by
  \begin{displaymath}
    B_{\norm{\cdot}}(r) = \left\{ y: \norm{y} \le r \right\},
  \end{displaymath}
  $\proj_C(x)$ is the projection operator defined by
  \begin{displaymath}
    \proj_C(x) = \mathop{\arg\min}\limits_{u \in C} \norm{u - x}_2.
  \end{displaymath}
  
  
  \item {Matrix $\ell_{1,2}$-norm:} $$h(x) = \norm{x}_{1,2} = 
  \sum_{i = 1}^{n} \norm{x(i,1:l)}_2, $$
  where $x \in \mr^{n \times l}$. The proximal operator of $\mu h(x)$ 
  is given by
  \begin{displaymath}
    (\prox_{\mu h}(x))_{i,j} = 
    \begin{cases}
      x_{i,j} \max\left\{ 1 - \frac{\mu}{\norm{x(i,1:l)}_2}, 0 
      \right\}, & \norm{x(i,1:l)}_2 > 0, \\
      0, & \norm{x(i,1:l)}_2 = 0.
    \end{cases}
  \end{displaymath}
  
  
  %\item {Matrix $\ell_{2,1}$-norm:} $$h(x) = \norm{x}_{2,1} = 
  %  \left( \sum_{i = 1}^{n} \norm{x(i,1:l)}_1^2 
  %\right)^{\frac{1}{2}}, 
  %  $$
  %  where $x \in \mr^{n \times l}$. The proximal operator of $\mu 
  %h(x)$ 
  %  is given by
\end{itemize}






\section{Numerical experiments}
\label{sec:num}




%\bibliographystyle{plain}
%\bibliography{ref}


%\appendix
%\renewcommand{\appendixname}{Appendix~\Alph{section}}
%
%\newpage
  
\end{document}
