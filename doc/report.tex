%!TEX program = xelatex

\documentclass[a4paper]{article}
%\documentclass{amsart}
\usepackage[breaklinks,linkcolor=black,citecolor=black,urlcolor=black]
{hyperref}
\usepackage{amsmath}
\usepackage{amssymb}
\usepackage{amsthm}
\usepackage{amsfonts}
\usepackage{booktabs}
\usepackage{graphicx}
\usepackage{latexsym}
\usepackage{mathrsfs}
\usepackage[ruled]{algorithm2e}
\usepackage{subfigure}

%\usepackage{algorithm}
%\usepackage{algorithmic}
\usepackage{algpseudocode}
\usepackage{enumerate}
\usepackage{bm}
\usepackage{listings}
\usepackage{tikz}
\usetikzlibrary{arrows}
\usepackage[colorinlistoftodos]{todonotes}
\usepackage[a4paper,left=2.5cm, right=2.5cm, top=2.5cm, 
bottom=2.5cm]{geometry}
\usepackage{xeCJK}
\usepackage{cleveref}

\newfontfamily{\con}{Consolas} 

%\lstset{language=Matlab}

\newcommand{\norm}[1]{\left\Vert#1\right\Vert}
\newtheorem{definition}{Definition}
\newtheorem{theorem}{Theorem}
\newtheorem{remark}{Remark}
\newtheorem{lemma}{Lemma}
\newtheorem{example}{Example}
\newtheorem{assumption}{Assumption}
\newtheorem{proposition}{Proposition}
\newtheorem{experiment}{Experiment}

%\numberwithin{equation}{section}
\newcommand{\ip}[2]{\left\langle{#1},{#2}\right\rangle} % inner-prod
\newcommand{\mr}{\mathbb{R}}
\newcommand{\mh}{\mathcal{H}}
\newcommand{\prox}{\mathrm{prox}}
\newcommand{\sign}{\mathrm{sign}}
\newcommand{\proj}{\mathrm{Proj}}
\newcommand{\mi}{\mathbf{1}}
\newcommand{\diag}{\mathrm{diag}}
\newcommand{\rank}{\mathrm{rank}}
\newcommand{\co}[1]{{\con{#1}}}

%\graphicspath{{../figures/}}



%--------------------------------------------------------------------%
\title{Software implementation for the proximal gradient methods}
\author{Zeyu Jin, Ting Lin}
\date{\today}

\begin{document}
  \maketitle
  \tableofcontents
  
  \section{Introduction}
  \label{sec:intro}
  In this project, we consider the following composite optimization 
  problem
  \begin{equation}
    \min\limits_{x} F(x) = f(x) + \mu h(x),
  \end{equation}
  where $f(x)$ is differentiable, $h(x)$ is a function whose 
  proximal operator is easily available, $\mu > 0$ is a given 
  parameter.
  The proximal operator of $\mu h(x)$ is defined as
  \begin{displaymath}
    \prox_{\mu h}(x) = \mathop{\arg\min}\limits_{u} \mu h(u) + 
    \frac{1}{2} \norm{u - x}^2.
  \end{displaymath}
  Starting from a suitable initial point $x^0$, the proximal gradient 
  method is performed as
  \begin{equation}\label{equ:pg}
    x^{k+1} = \prox_{t_k h} (x^k - t_k \nabla f(x^k)),
  \end{equation}
  where $t_k$ is a chosen step size.
  
  Our contributions are listed as follows:
  \begin{itemize}
    \item 
  \end{itemize}

  The rest of this report is organized as follows. In \cref{sec:alg}, 
  we introduce some details of our algorithms. In \cref{sec:cal}, we 
  present several typical scenarios of $f(x)$ and $h(x)$. In 
  \cref{sec:num}, we conduct several numerical experiments. Detailed 
  description of the design of the packages are presented in 
  \cref{app:design}.
  
  
  
  % description of algorithms & interpretation of implementation?
  \section{Algorithms}
  \label{sec:alg}
  
  \subsection{Strategies for choosing the step size}
  In \cref{sec:intro}, we have already given the basic frame of the 
  proximal gradient method. One may notice from \cref{equ:pg} that we 
  need to determine the step size $t_k$ at each step. In our 
  implementations, the following four types of strategies for 
  choosing the step size are supported.
  \begin{itemize}
    \item \co{Constant}: constant step size: $t_k = t_0, \ \forall k 
    \in \mathbb{N}$.
    \item \co{Armijo}: backtracking line search to achieve the Armijo 
    condition. The step size $t_k$ is chosen such that
    \begin{displaymath}
      f(x^{k+1}) \le f(x^k) + \rho \ip{\nabla f(x^k)}{x^{k+1} - x^k},
    \end{displaymath}
    where $\rho \in (0, 1)$ is given beforehand.
    \item \co{Nonmonotone}: backtracking line search to achieve a 
    non-monotone condition using the BB step size. The step size 
    $t_k$ is chosen such that 
    \begin{displaymath}
      f(x^{k+1}) \le \max\left\{
      f(x^j) : j = \max\{0, k-M+1\},\ldots,k \right\} + \rho 
      \ip{\nabla f(x^k)}{x^{k+1} - x^k},
    \end{displaymath}
    where $\rho \in (0, 1)$ and $M > 0$ are given beforehand.
    \item \co{Classical}: a classical backtracking line search 
    strategy to choose the step size in the proximal gradient method. 
    The step size $t_k$ is chosen such that 
    \begin{displaymath}
      f(x^{k+1}) \le f(x^k) + \ip{\nabla f(x^k)}{x^{k+1}- x^k} + 
      \frac{1}{2t_k} \norm{x^{k+1} - x^k}^2.
    \end{displaymath}
  \end{itemize}
  For the last three strategies, we use the we adopt the Armijo 
  backtracking method and adopt BB rule to determine the initial 
  guess of the step size. 

  \subsection{Continuation strategy}
  Later, we may use the continuation strategy.
  The basic scheme of this strategy is presented in Algorithm 
  \ref{alg:cont}. This strategy may improve efficiency for some 
  particular types of $h(x)$.
  \begin{algorithm}[h]
    \caption{Continuation strategy.}
    \label{alg:cont}
    \LinesNumbered
    Given the factor $\eta \in (0,1)$, the initial value $\mu_0$ and 
    the starting point $x_0$. Let $t \leftarrow 0$.\\
    \While{$\mu_t \ne \mu$ or {the stopping criteria are not 
        satisfied}}{
      Solve the group LASSO problem with parameter $\mu_t$ and 
      initial 
      value $x_t$, and obtain the optimal point $x_{t+1}$.\\
      $\mu_{t + 1} \leftarrow \max\{\eta \mu_t, \mu\}$.\\
      $t \leftarrow t + 1$.
    }
  \end{algorithm}
  
  
  
  
  
  \section{Several scenarios}
  \label{sec:cal}
  
  \subsection{Scenarios of $f(x)$}
  \label{sec:cal_f}
  In this part, we list several typical scenarios of $f(x)$ that are 
  supported in our codes. We give their gradient explicitly.
  
  \begin{itemize}
    \item {Least squares in vector form:} $f(x) = \frac{1}{2} 
    \norm{Ax - b}_2^2$, where $A \in \mr^{m \times n}$, $x \in 
    \mr^n$, 
    $b 
    \in \mr^m$. The gradient of $f(x)$ is given by %\footnote{In this 
    %report, we write the gradient in the same shape as $x$.}
    \begin{displaymath}
      \nabla f(x) = A^T (A x - b).
    \end{displaymath}
    
    \item {Least squares in matrix form:} $f(x) = \frac{1}{2} 
    \norm{Ax - b}_F^2$, where $A \in \mr^{m \times n}$, $x \in \mr^{n 
      \times l}$, $b \in \mr^{m \times l}$. The gradient of $f(x)$ is 
    given 
    by
    \begin{displaymath}
      \nabla f(x) = A^T  (A x - b).
    \end{displaymath}
    
    \item {Logistic regression:} 
    \begin{displaymath}
      f(x) = \frac{1}{m} \sum_{i = 1}^{m} \log(1 + \exp(- b_i a_i^T 
      x)),
    \end{displaymath}
    where $x \in \mr^n$, $A = (a_1, \ldots, a_m) \in \mr^{n \times 
      m}$, 
    $a_i = (a_{1,i}, \ldots, a_{n,i})^T \in \mr^n$,
    $b = (b_1, \ldots, b_m)^T \in \mr^m$. The gradient of $f(x)$ is 
    given 
    by
    \begin{displaymath}
      \nabla f(x) = - \frac{1}{m} \sum_{i = 1}^{m} \frac{b_i 
        a_i}{\exp(b_i a_i^T x) + 1}.
    \end{displaymath}
  \end{itemize}
  
  
  
  \subsection{Scenarios of $h(x)$}
  \label{sec:cal_h}
  In this part, we list several typical scenarios of $h(x)$ that are 
  supported in our codes. We give the proximal operator of $\mu h(x)$ 
  explicitly.
  
  \begin{itemize}
    \item {Vector $\ell_0$-norm:} $h(x) = \norm{x}_0 = \# \{i:x_i 
    \ne 0\}$, where $x \in \mr^n$. The proximal operator of $\mu 
    h(x)$ 
    is 
    given by
    \begin{displaymath}
      \prox_{\mu h}(x) = \mh_{\mu}(x_1) \times \ldots \times 
      \mh_{\mu}(x_n),
    \end{displaymath}
    where $\mh_{\mu}(s)$ is given by
    \begin{displaymath}
      \mh_{\mu}(s) = 
      \begin{cases}
        \{0\},    & |s| < \sqrt{2\mu}, \\
        \{0, s\}, & |s| = \sqrt{2\mu}, \\
        \{s\},    & |s| > \sqrt{2\mu}.
      \end{cases}
    \end{displaymath}
    
    \item {Vector $\ell_1$-norm:} $h(x) = \norm{x}_1 = 
    \sum_{i=1}^{n} |x_i|$, where $x \in \mr^n$. The proximal operator 
    of 
    $\mu h(x)$ is given by
    \begin{displaymath}
      \prox_{\mu h}(x) = (\mh_{\mu}(x_1), \ldots, \mh_{\mu}(x_n))^T,
    \end{displaymath}
    where $\mh_{\mu}(s)$ is given by
    \begin{displaymath}
      \mh_{\mu}(s) = \sign(s) \max\{ |s| - \mu, 0 \}.
    \end{displaymath}
    
    \item {Vector $\ell_2$-norm:} $h(x) = \norm{x}_2 = 
    \left( \sum_{i=1}^{n} |x_i|^2 \right)^{\frac{1}{2}}$, where $x 
    \in 
    \mr^n$. The proximal operator of $\mu h(x)$ is given by
    \begin{displaymath}
      \prox_{\mu h}(x) = 
      \begin{cases}
        \frac{\max\{ \norm{x}_2 - \mu, 0 \}}{\norm{x}_2} x, & x \ne 
        0, 
        \\
        0, & x = 0.
      \end{cases}
    \end{displaymath}
    
    
    \item {Vector $\ell_\infty$-norm:} $h(x) = \norm{x}_\infty = 
    \max\{ |x_i|: i = 1, 2, \ldots, n \}$, where $x \in 
    \mr^n$. By Moreau's decomposition, the proximal operator of $\mu 
    h(x)$ is given by
    \begin{displaymath}
      \begin{split}
        \prox_{\mu h}(x) & = x - \mu \; \prox_{\mu^{-1} h^\star} 
        \left( 
        \frac{x}{\mu} \right) \\
        & = x - \mu \; \proj_{B_{\norm{\cdot}_1}(1)} \left( 
        \frac{x}{\mu} \right),
      \end{split}
    \end{displaymath}
    where $h^\star(y) = \mi_{B_{\norm{\cdot}_1}(1)}(y)$, $\mi_C(y)$ 
    is 
    the indicator function defined by
    \begin{displaymath}
      \mi_C(y) = 
      \begin{cases}
        0, & y \in C, \\
        +\infty, & y \notin C,
      \end{cases}
    \end{displaymath}
    $B_{\norm{\cdot}}(r)$ is the norm ball defined by
    \begin{displaymath}
      B_{\norm{\cdot}}(r) = \left\{ y: \norm{y} \le r \right\},
    \end{displaymath}
    $\proj_C(x)$ is the projection operator defined by
    \begin{displaymath}
      \proj_C(x) = \mathop{\arg\min}\limits_{u \in C} \norm{u - x}_2.
    \end{displaymath}
    
    
    \item {Matrix $\ell_{1,2}$-norm:} $$h(x) = \norm{x}_{1,2} = 
    \sum_{i = 1}^{n} \norm{x(i,1:l)}_2, $$
    where $x \in \mr^{n \times l}$. The proximal operator of $\mu 
    h(x)$ 
    is given by
    \begin{displaymath}
      (\prox_{\mu h}(x))_{i,j} = 
      \begin{cases}
        x_{i,j} \max\left\{ 1 - \frac{\mu}{\norm{x(i,1:l)}_2}, 0 
        \right\}, & \norm{x(i,1:l)}_2 > 0, \\
        0, & \norm{x(i,1:l)}_2 = 0.
      \end{cases}
    \end{displaymath}
    
    
    %\item {Matrix $\ell_{2,1}$-norm:} $$h(x) = \norm{x}_{2,1} = 
    %  \left( \sum_{i = 1}^{n} \norm{x(i,1:l)}_1^2 
    %\right)^{\frac{1}{2}}, 
    %  $$
    %  where $x \in \mr^{n \times l}$. The proximal operator of $\mu 
    %h(x)$ 
    %  is given by
    
    \item {Nuclear norm:} $h(x) = \norm{x}_* = \sum_{i = 1}^{r} 
    \sigma_i(x)$, where $x \in \mr^{m \times n}$, $\{ \sigma_i(x) 
    \}_{i=1}^r$ are all the $r$ non-zero singular values of $x$. Let 
    the SVD of $x$ be $x = U \Sigma V^T$, where $\Sigma = 
    \diag_{m,n}\left\{ \sigma_1, \ldots, \sigma_r \right\}$ denotes 
    the 
    $m \times n$ matrix $\Sigma$ with $\Sigma_{i,i} = \sigma_i$ for 
    $i 
    = 1, 2, \ldots, r$ and other elements zero. Here $\sigma_1 \ge 
    \sigma_2 \ge \ldots \ge \sigma_r$. Let $d_i = \max\{ 
    \sigma_i - \mu, 0 \}$ and $D = \diag_{m,n}\left\{ d_1, \ldots, 
    d_r 
    \right\}$. The proximal operator of $\mu h(x)$ is given by
    \begin{displaymath}
      \prox_{\mu h}(x) = U D V^T.
    \end{displaymath}
    
    \item {Elastic-net:} $h(x) = \norm{x}_1 + \frac{\lambda}{2} 
    \norm{x}_2^2$, where $x \in \mr^n$. The proximal operator of $\mu 
    h(x)$ is given by
    \begin{displaymath}
      (\prox_{\mu h}(x))_i =  \frac{\sign(y_i) \; \max\{ |y_i| - \mu, 
        0 
        \}}{1 + \lambda \mu}.
    \end{displaymath}
    
    \item {Sum of logarithms:} $h(x) = -\sum_{i=1}^{n} \log(x_i)$, 
    where $x \in \mr^n$. The proximal operator of $\mu h(x)$ 
    is given by
    \begin{displaymath}
      (\prox_{\mu h}(x))_i = \frac{x_i + \sqrt{x_i^2 + 4\mu}}{2}.
    \end{displaymath}
    
    \item {Indicator of $\ell_0$-ball:} $h(x) = 
    \mi_{B_{\norm{\cdot}_0}(R)}$, where $x \in \mr^n$ and $R$ is a 
    positive integer. Let $\left\{ \sigma(1), \sigma(2), \ldots, 
    \sigma(n) \right\}$ be a permutation of $\left\{  1, 2, \ldots, n 
    \right\}$ such that 
    \begin{displaymath}
      |x_{\sigma(1)}| \ge |x_{\sigma(2)}| \ge \ldots \ge 
      |x_{\sigma(n)}|.
    \end{displaymath}
    The proximal operator of $\mu h(x)$ is given by
    \begin{displaymath}
      (\prox_{\mu h}(x))_{\sigma(i)} = 
      \begin{cases}
        x_{\sigma(i)}, & i = 1, 2, \ldots, R, \\
        0, & i = R + 1, R + 2, \ldots, 
        n.
      \end{cases}
    \end{displaymath}
    
    \item {Indicator of $\ell_2$-ball:} $h(x) = 
    \mi_{ B_{\norm{\cdot}_2}(R)}$, where $x, y \in \mr^n$.
    The proximal operator of $\mu h(x)$ is given by
    \begin{displaymath}
      \prox_{\mu h}(x) = 
      \begin{cases}
        x, & \norm{x}_2 \le R, \\
        \frac{R}{\norm{x}_2}(x), & \norm{x}_2 > R.
      \end{cases}
    \end{displaymath}
    
    \item {Indicator of simple box:} $$h(x) = \mi_{[l_1,u_1] \times 
      \ldots \times [l_n,u_n]}(x),$$
    where $x \in \mr^n$. Here $l_i$ and $u_i$ are allowed to be 
    $-\infty$ and $+\infty$, respectively. The proximal operator of 
    $\mu h(x)$ is given by
    \begin{displaymath}
      (\prox_{\mu h}(x))_i = \min\{ u_i, \max\{l_i, x_i\} \}.
    \end{displaymath}
    
    \item {Indicator of rank ball:} $h(x) = \mi_{\rank(\cdot) \le 
      R}$, 
    where $x \in \mr^{m \times n}$ and $R$ is a positive integer. Let 
    the SVD of $x$ be $x = U \Sigma V^T$, where $\Sigma = 
    \diag_{m,n}\left\{ \sigma_1, \ldots, \sigma_r \right\}$. Let 
    \begin{displaymath}
      d_i = 
      \begin{cases}
        \sigma_i, & i = 1, 2, \ldots, R, \\
        0, & i = R+1, R+2, \ldots, r,
      \end{cases}
    \end{displaymath}
    and $D = \diag_{m,n}\left\{ d_1, \ldots, d_r \right\}$.
    The proximal operator of $\mu h(x)$ is given by
    \begin{displaymath}
      \prox_{\mu h}(x) = U D V^T.
    \end{displaymath}
    
  \end{itemize}
  
  
  
  
  
  
  \section{Numerical experiments}
  \label{sec:num}
  
  
  
  
  \begin{example}\label{exp1}
    In this numerical example, we test the classical LASSO problem.
    In other words, the types of $f(x)$ and $h(x)$ are \co{LS} and 
    \co{L1\_NORM}, respectively. Let the elements of $A \in \mr^{m 
    \times n}$ and $u \in \mr^n$ be uniformly random numbers from 
    $[0,1]$. Let $b = Au$. We take $m = 256$ and $n = 512$. The 
    initial value is $x^0 = 0$. Let $\mu = 0.01$. We use continuation 
    strategy and compare between three 
    types of line search methods. The numerical results are presented 
    in \cref{tab1}.
  \end{example}
  
  \begin{table}[p]
    \centering
    \begin{tabular}{cc|ccccc}
      \hline
      type & cont. & iters & cputime & 
      fval & optimality & flag \\ \hline
      Armijo & 1 & 135 & 0.09375 & 0.223574 & 0.00367652 & 1 \\
      Nonmonotone & 1 & 114 & 0.046875 & 0.223573 & 2.34637e-05 & 1 \\
      Classical & 1 & 109 & 0.046875 & 0.223573 & 3.50546e-05 & 1 \\
      \hline
    \end{tabular}
    \caption{Numerical results for \cref{exp1}.}
    \label{tab1}
  \end{table}

  \begin{example}\label{exp2}
    In this numerical example, we test the group LASSO problem.
    In other words, the types of $f(x)$ and $h(x)$ are \co{LS} and 
    \co{L12\_NORM}, respectively. Let the elements of $A \in \mr^{m 
    \times n}$ and $u \in \mr^{n \times l}$ be uniformly random 
    numbers from $[0,1]$. Let $b = Au$. We take $m = 128$, $n = 256$ 
    and $l = 2$. The initial value is $x^0 = 0$. Let $\mu = 0.01$. We 
    use continuation strategy and compare between three 
    types of line search methods. The numerical results are presented 
    in \cref{tab2}.
  \end{example}
  
  \begin{table}[p]
    \centering
    \begin{tabular}{cc|ccccc}
      \hline
      type & cont. & iters & cputime & 
      fval & optimality & flag \\ \hline
      Armijo & 1 & 198 & 0.3125 & 0.178611 & 0.0218023 & 1 \\
      Nonmonotone & 1 & 327 & 0.28125 & 0.173687 & 0.000139197 & 1 \\
      Classical & 1 & 334 & 0.28125 & 0.173687 & 5.51511e-05 & 1 \\
      \hline
    \end{tabular}
    \caption{Numerical results for \cref{exp2}.}
    \label{tab2}
  \end{table}

  \begin{example}\label{exp3}
%    In this numerical example, we test the group LASSO problem.
%    In other words, the types of $f(x)$ and $h(x)$ are \co{LS} and 
%    \co{L12\_NORM}, respectively. Let the elements of $A \in \mr^{m 
%      \times n}$ and $u \in \mr^{n \times l}$ be uniformly random 
%    numbers from $[0,1]$. Let $b = Au$. We take $m = 128$, $n = 256$ 
%    and $l = 2$. The initial value is $x^0 = 0$. Let $\mu = 0.01$. 
%We 
%    use continuation strategy and compare between three 
%    types of line search methods. The numerical results are 
%presented 
%    in \cref{tab3}.
  \end{example}
  
  \begin{table}[p]
    \centering
    \begin{tabular}{cc|ccccc}
      \hline
      type & cont. & iters & cputime & 
      fval & optimality & flag \\ \hline
      Armijo & 1 & 29 & 0 & 4.86937 & 0.0201708 & 1 \\
      Nonmonotone & 1 & 37 & 0 & 4.86934 & 0.000387912 & 1 \\
      Classical & 1 & 54 & 0 & 4.86934 & 0.000220955 & 1 \\
      \hline
    \end{tabular}
    \caption{Numerical results for \cref{exp3}.}
    \label{tab3}
  \end{table}

\begin{example}\label{exp4}
  %    In this numerical example, we test the group LASSO problem.
  %    In other words, the types of $f(x)$ and $h(x)$ are \co{LS} and 
  %    \co{L12\_NORM}, respectively. Let the elements of $A \in 
  %\mr^{m 
  %      \times n}$ and $u \in \mr^{n \times l}$ be uniformly random 
  %    numbers from $[0,1]$. Let $b = Au$. We take $m = 128$, $n = 
  %256$ 
  %    and $l = 2$. The initial value is $x^0 = 0$. Let $\mu = 0.01$. 
  %We 
  %    use continuation strategy and compare between three 
  %    types of line search methods. The numerical results are 
  %presented 
  %    in \cref{tab3}.
\end{example}

\begin{table}[p]
  \centering
  \begin{tabular}{cc|ccccc}
    \hline
    type & cont. & iters & cputime & 
    fval & optimality & flag \\ \hline
    Armijo & 1 & 258 & 0 & 0.0258146 & 7.61302e-05 & 1 \\
    Nonmonotone & 1 & 666 & 0 & 0.0253084 & 2.88564e-05 & 1 \\
    Classical & 1 & 792 & 0 & 0.0253074 & 3.99053e-06 & 1 \\
    \hline
  \end{tabular}
  \caption{Numerical results for \cref{exp4}.}
  \label{tab4}
\end{table}

\begin{example}\label{exp5}
  %    In this numerical example, we test the group LASSO problem.
  %    In other words, the types of $f(x)$ and $h(x)$ are \co{LS} and 
  %    \co{L12\_NORM}, respectively. Let the elements of $A \in 
  %\mr^{m 
  %      \times n}$ and $u \in \mr^{n \times l}$ be uniformly random 
  %    numbers from $[0,1]$. Let $b = Au$. We take $m = 128$, $n = 
  %256$ 
  %    and $l = 2$. The initial value is $x^0 = 0$. Let $\mu = 0.01$. 
  %We 
  %    use continuation strategy and compare between three 
  %    types of line search methods. The numerical results are 
  %presented 
  %    in \cref{tab3}.
\end{example}

\begin{table}[p]
  \centering
  \begin{tabular}{cc|ccccc}
    \hline
    type & cont. & iters & cputime & 
    fval & optimality & flag \\ \hline
    Armijo & 1 & 66 & 0 & 0.0523895 & 0.000105485 & 1 \\
    Nonmonotone & 1 & 68 & 0 & 0.0523895 & 0.000147343 & 1 \\
    Classical & 1 & 81 & 0 & 0.0523896 & 0.000307073 & 1 \\
    \hline
  \end{tabular}
  \caption{Numerical results for \cref{exp5}.}
  \label{tab5}
\end{table}
  
  
  \begin{example}\label{exp5}
    %    In this numerical example, we test the group LASSO problem.
    %    In other words, the types of $f(x)$ and $h(x)$ are \co{LS} 
    %and 
    %    \co{L12\_NORM}, respectively. Let the elements of $A \in 
    %\mr^{m 
    %      \times n}$ and $u \in \mr^{n \times l}$ be uniformly 
    %random 
    %    numbers from $[0,1]$. Let $b = Au$. We take $m = 128$, $n = 
    %256$ 
    %    and $l = 2$. The initial value is $x^0 = 0$. Let $\mu = 
    %0.01$. 
    %We 
    %    use continuation strategy and compare between three 
    %    types of line search methods. The numerical results are 
    %presented 
    %    in \cref{tab3}.
  \end{example}
  
  \begin{table}[p]
    \centering
    \begin{tabular}{cc|ccccc}
      \hline
      type & cont. & iters & cputime & 
      fval & optimality & flag \\ \hline
      Armijo & 1 & 66 & 0 & 0.0523895 & 0.000105485 & 1 \\
      Nonmonotone & 1 & 68 & 0 & 0.0523895 & 0.000147343 & 1 \\
      Classical & 1 & 81 & 0 & 0.0523896 & 0.000307073 & 1 \\
      \hline
    \end{tabular}
    \caption{Numerical results for \cref{exp5}.}
    \label{tab5}
  \end{table}
  
  
  \begin{example}\label{exp6}
    %    In this numerical example, we test the group LASSO problem.
    %    In other words, the types of $f(x)$ and $h(x)$ are \co{LS} 
    %and 
    %    \co{L12\_NORM}, respectively. Let the elements of $A \in 
    %\mr^{m 
    %      \times n}$ and $u \in \mr^{n \times l}$ be uniformly 
    %random 
    %    numbers from $[0,1]$. Let $b = Au$. We take $m = 128$, $n = 
    %256$ 
    %    and $l = 2$. The initial value is $x^0 = 0$. Let $\mu = 
    %0.01$. 
    %We 
    %    use continuation strategy and compare between three 
    %    types of line search methods. The numerical results are 
    %presented 
    %    in \cref{tab3}.
  \end{example}
  
  \begin{table}[p]
    \centering
    \begin{tabular}{cc|ccccc}
      \hline
      type & cont. & iters & cputime & 
      fval & optimality & flag \\ \hline
      Armijo & 1 & 3 & 0 & 1.88988 & 2.05963e-08 & 1 \\
      Nonmonotone & 1 & 3 & 0 & 1.88988 & 2.05963e-08 & 1 \\
      Classical & 1 & 10 & 0 & 1.88988 & 2.67976e-09 & 1 \\
      \hline
    \end{tabular}
    \caption{Numerical results for \cref{exp6}.}
    \label{tab6}
  \end{table}

\begin{example}\label{exp7}
  %    In this numerical example, we test the group LASSO problem.
  %    In other words, the types of $f(x)$ and $h(x)$ are \co{LS} 
  %and 
  %    \co{L12\_NORM}, respectively. Let the elements of $A \in 
  %\mr^{m 
  %      \times n}$ and $u \in \mr^{n \times l}$ be uniformly 
  %random 
  %    numbers from $[0,1]$. Let $b = Au$. We take $m = 128$, $n = 
  %256$ 
  %    and $l = 2$. The initial value is $x^0 = 0$. Let $\mu = 
  %0.01$. 
  %We 
  %    use continuation strategy and compare between three 
  %    types of line search methods. The numerical results are 
  %presented 
  %    in \cref{tab3}.
\end{example}

\begin{table}[p]
  \centering
  \begin{tabular}{cc|ccccc}
    \hline
    type & cont. & iters & cputime & 
    fval & optimality & flag \\ \hline
    Armijo & 1 & 74 & 0.546875 & 815.136 & 0.00171295 & 1 \\
    Nonmonotone & 1 & 151 & 0.703125 & 815.136 & 0.000385093 & 1 \\
    Classical & 1 & 191 & 0.96875 & 815.136 & 0.000632897 & 1 \\
    \hline
  \end{tabular}
  \caption{Numerical results for \cref{exp7}.}
  \label{tab7}
\end{table}
  
  
  
  \begin{example}\label{exp8}
    %    In this numerical example, we test the group LASSO problem.
    %    In other words, the types of $f(x)$ and $h(x)$ are \co{LS} 
    %and 
    %    \co{L12\_NORM}, respectively. Let the elements of $A \in 
    %\mr^{m 
    %      \times n}$ and $u \in \mr^{n \times l}$ be uniformly 
    %random 
    %    numbers from $[0,1]$. Let $b = Au$. We take $m = 128$, $n = 
    %256$ 
    %    and $l = 2$. The initial value is $x^0 = 0$. Let $\mu = 
    %0.01$. 
    %We 
    %    use continuation strategy and compare between three 
    %    types of line search methods. The numerical results are 
    %presented 
    %    in \cref{tab3}.
  \end{example}
  
  \begin{table}[p]
    \centering
    \begin{tabular}{cc|ccccc}
      \hline
      type & cont. & iters & cputime & 
      fval & optimality & flag \\ \hline
      Armijo & 1 & 14 & 0 & 2016.87 & 1.01978e-05 & 1 \\
      Nonmonotone & 1 & 14 & 0 & 2016.87 & 1.01978e-05 & 1 \\
      Classical & 1 & 14 & 0 & 2016.87 & 1.01978e-05 & 1 \\
      \hline
    \end{tabular}
    \caption{Numerical results for \cref{exp8}.}
    \label{tab8}
  \end{table}

\begin{example}\label{exp9}
  %    In this numerical example, we test the group LASSO problem.
  %    In other words, the types of $f(x)$ and $h(x)$ are \co{LS} 
  %and 
  %    \co{L12\_NORM}, respectively. Let the elements of $A \in 
  %\mr^{m 
  %      \times n}$ and $u \in \mr^{n \times l}$ be uniformly 
  %random 
  %    numbers from $[0,1]$. Let $b = Au$. We take $m = 128$, $n = 
  %256$ 
  %    and $l = 2$. The initial value is $x^0 = 0$. Let $\mu = 
  %0.01$. 
  %We 
  %    use continuation strategy and compare between three 
  %    types of line search methods. The numerical results are 
  %presented 
  %    in \cref{tab3}.
\end{example}

\begin{table}[p]
  \centering
  \begin{tabular}{cc|ccccc}
    \hline
    type & cont. & iters & cputime & 
    fval & optimality & flag \\ \hline
    Armijo & 1 & 42 & 0 & 0.00905712 & 0.00505199 & 1 \\
    Nonmonotone & 1 & 77 & 0 & 0.00849407 & 0.00280368 & 1 \\
    Classical & 1 & 102 & 0 & 0.00845289 & 9.38481e-05 & 1 \\
    \hline
  \end{tabular}
  \caption{Numerical results for \cref{exp9}.}
  \label{tab9}
\end{table}
  
  
  %\bibliographystyle{plain}
  %\bibliography{ref}
  
  
  \appendix
  \renewcommand{\appendixname}{Appendix~\Alph{section}}
  
  %\newpage
  
  \section{Design of the packages}
  \label{app:design}
  
\end{document}
